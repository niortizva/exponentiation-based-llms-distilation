\documentclass[12pt]{article}
\usepackage[
  a4paper
]{geometry}
\usepackage[utf8]{inputenc}
\usepackage[english]{babel}

%% Aditional Packages
\usepackage{url}
\usepackage{graphicx}
\usepackage{subcaption}
\usepackage{tikz}
\usetikzlibrary{shapes, arrows.meta, fit, automata, positioning}
\usepackage{amsmath}
\usepackage{amsthm}
\usepackage{amssymb}
\usepackage{amsfonts}
\usepackage[natbibapa]{apacite}
%% \usepackage[OT1]{fontenc}
%% \usepackage{nopageno}
\usepackage{hyperref}

%% Commands
%% \setlength{\parindent}{0pt}
\newcommand\defeq{\stackrel{\mathtt{def}}{=}}

% Important to explain some requirements or interesting facts
\newtheorem{remark}{Remark}
% Definition: an explanation of the mathematical meaning of a word.
\newtheorem{definition}{Definition}
% Proposition: a statement of fact that is true and interesting in a given context.
\newtheorem{proposition}{Proposition}
% Lemma: a true statement used in proving other true statements.
\newtheorem{lemma}{Lemma}
% Corollary: a true statement that is a simple deduction from a theorem or proposition.
\newtheorem{corollary}{Corollary}
% Example: an illustration of a definition or theorem.
\newtheorem{example}{Example}

\newcommand{\vectorspace}[2]{\mathbb{R}^{#1 \times #2}}
\newcommand{\I}[2]{I_{#1 \times #2}}
\newcommand{\0}[2]{\mathbf{0}_{#1 \times #2}}
\newcommand\A{\mathbf{A}}
\newcommand\B{\mathbf{B}}
\newcommand\C{\mathbf{C}}
\newcommand\K{\mathbf{K}}
\newcommand\M{\mathbf{M}}
\newcommand\N{\mathbf{N}}
\newcommand\Q{\mathbf{Q}}
\newcommand\U{\mathbf{U}}
\newcommand\V{\mathbf{V}}
\newcommand\X{\mathbf{X}}
\newcommand\W{\mathbf{W}}
\newcommand\R{\mathbb{R}}
\newcommand\z{\mathbf{z}}
\newcommand\x{\times}
\newcommand\softmax{softmax}
\newcommand{\vc}[1]{\mathbf{#1}}
\newcommand{\mx}[1]{\mathbf{\uppercase{#1}}}
\newcommand{\norm}[1]{\| #1 \|}
\newcommand{\expo}[1]{\mathit{e}^{#1}}

\begin{document}

\title{Non-square Matrix Product Exponentiation-based Attention}
\author{Ortiz Valencia, Nicolás}
\date{}

\maketitle

\section{Introduction}

Polynomial Neural Networks (PNNs) demonstrate the capability of learning complex feature representations through hierarchical function composition. Concurrently, the multi-head attention mechanism has been established as a cornerstone of modern Large Language Models (LLMs), enabling them to capture diverse contextual relationships. However, the computational and parametric cost of multi-head attention presents a significant bottleneck for resource-constrained applications. This work investigates a novel exponentiation-based attention approach, inspired by the functional form of multi-head attention, with the goal of capturing rich, high-order feature interactions without the prohibitive cost associated with standard LLM architectures.
\section{Background and Rationale}

Transformer architectures, built on attention mechanisms, are now fundamental to modern deep learning, particularly in the fields of natural language processing and computer vision. In particular, the multi-head attention mechanism, as introduced in the seminal work by \cite{Vaswani2017}, allows models to focus on different parts of the input data simultaneously, enhancing their ability to capture complex patterns and dependencies. Nonetheless, pre-trained attention-based models are computationally inefficient, require substantial amounts of data, and need to be fine-tuned for specific tasks \citep{Lin2022}. \\

In this section, we provide a concise overview of the standard attention mechanism, emphasizing its mathematical formulation and the role of exponentiation within it. This sets the stage for our proposed methodology, which seeks to redefine attention through the lens of non-square matrix exponentiation.

\subsection{The standard attention mechanism}

The standard attention mechanism 

operates by projecting an input matrix into multiple subspaces via parallel "heads." For a given head $i$, the operation is defined as:

\begin{equation}
  \label{eq:multi-head-attention}
  H_{i} \defeq softmax \left(\frac{\Q_i \K_i^T }{ \sqrt{d_{k}}}\right) \V_i,
\end{equation}

where the constituent matrices are derived from the input $\X \in \mathbb{R}^{m \times n}$ as follows:

\begin{align*}
  \Q_i &= \X \W_{q_i}, \quad \W_{q_i} \in \mathbb{R}^{n \times d_k}, \\
  \K_i &= \X \W_{k_i}, \quad \W_{k_i} \in \mathbb{R}^{n \times d_k}, \\
  \V_i &= \X \W_{v_i}, \quad \W_{v_i} \in \mathbb{R}^{n \times d_v}.
\end{align*}

Here, $\Q_i, \K_i \in \mathbb{R}^{m \times d_k}$ are the query and key matrices, respectively, and $\V_i \in \mathbb{R}^{m \times d_v}$ is the value matrix.\\

The core of the attention mechanism lies in the scaled dot-product $\Q_i \K_i^T$. The subsequent application of the $\softmax$ function, defined for a vector $\z \in \mathbb{R}^K$ as $\sigma(\z)_i = \exp(z_i) / \sum_{j=1}^{K} \exp(z_j)$, imparts an exponential character to the entire operation. This observation is critical: the output of each attention head $H_i \in \mathbb{R}^{m \times d_v}$ is fundamentally a product of an exponentially-weighted matrix and a linear projection of the input as showed in Equation~\ref{eq:multi-head-attention-example}.\\

\begin{equation}
  \label{eq:multi-head-attention-example}
  H_{i} \defeq \left(\begin{matrix}
    \frac{\exp^z_{1, 1}}{ \sum_{j=1}^{m} \exp^z_{1, j}} & \cdots & \frac{\exp^z_{1, m}}{ \sum_{j=1}^{m} \exp^z_{1, j}} \\
    \vdots & \ddots & \vdots \\
    \frac{\exp^z_{m, 1}}{ \sum_{j=1}^{m} \exp^z_{m, j}} & \cdots & \frac{\exp^z_{m, m}}{ \sum_{j=1}^{m} \exp^z_{m, j}}
  \end{matrix}\right) \V_i,
\end{equation}

Finally, the model then concatenates all the outputs and projects them back to a $m$-dimensional representation as follows:

\begin{equation}
  MHA(\Q, \K, \V) = \text{concat}(H_1, \ldots, H_h) \W_o,
\end{equation}

where $\W_o \in \mathbb{R}^{d_v \times d_m}$. In essence, the attention mechanisms select, modulate, and focus on the information most relevant to behavior. According to \citet{deSantanaCorreia2022}, attention has existed for at least three decades and has been applied in various domains, including computer vision, natural language processing, and speech recognition, among others. Modern approaches to attention mechanisms are primarily based on the Transformer architecture \citep{Vaswani2017}, which relies on self-attention to model relationships between different parts of the input data. However, as mentioned earlier, the computational and parametric cost present significant challenges, and in response, other attention variants have been proposed, such as sparse attention, linear attention, prototype and memory Compression, low-rank attention, and attention with prior \citep{Lin2022}. 

Although several variants of attention mechanisms exist, the multi-head attention mechanism remains the most widely adopted due to its effectiveness in capturing diverse patterns~\citep{Bahdanau2015Rnn,Bahdanau2015Seq,Xiong2025,Zhang2025,Liu2025,Li2025,Sharifi2025,Hasan2024}. \\ 

\subsection{Attention Variants Taxonomy}

\subsubsection{Sparse Attention}

Sparse attention mechanisms aim to reduce the computational burden of standard attention by limiting the number of interactions between tokens. Techniques such as local attention, where each token attends only to its neighboring tokens, and global attention, which allows certain tokens to attend to all others, are common strategies~\citep{child2019generatinglongsequencessparse,Xu2021,Guo2019}. These methods significantly decrease the number of computations required, making them more efficient for long sequences.

\subsubsection{Linearized atttention}

Let $\Q,\K,\V \in \mathbb{R}^{n \times d_k}$, the complexity of computing $\softmax (\Q \K^T) \V$ is quadratic, then we can try to disentangle the equation into $\Q' \K'^{T}$, and we can compute $\Q \K^{T}$ in reverse order ($\Q (\K^T \V)$) leading to linear complexity. This is the main idea behind linearized attention mechanisms, which approximate the attention computation using kernel methods or low-rank approximations~\citep{Katharopoulos2020,Choromanski2022,Feng2022}. This could be especially beneficial for autoregressive attention, where the model generates sequences token by token, as it allows for efficient computation without sacrificing performance, and also enables Transformer decoders to run like RNNs~\citep{Lin2022} 

\subsubsection{Query Prototyping and Memory Compression}

Apart from the variants mentioned above, other approaches focus on compressing the attention mechanism itself by reducing the number of queries or keys used in the attention computation. Either the queries are selected from a subset of representative tokens (prototyping) or the keys and values are compressed into a smaller set of memory slots. These methods aim to retain the most salient information while reducing the overall computational load. They can lower the quadratic complexity of self-attention to linear or near-linear scaling, enabling more efficient processing of long inputs without substantial loss in performance.

\subsubsection{Low-Rank Attention}

Consider the attention matrix $\A = \softmax \left( \frac{\Q \K^{T}}{\sqrt(d_k)} \right)$. Theoretical analysis and empirical evidence suggest that $\A$ often exhibits low-rank properties \citep{Guo2019rank}, this means that the matrix can be approximated by a product of two smaller matrices, $\A = \U \V^{T}$, where $\U \in \mathbb{R}^{m \times r}$ and $\V \in \mathbb{R}^{n \times r}$ with $r << min(m,n)$. Decomposing the attention matrix into smaller components reduces the number of parameters and computation cost while preserving complex relationships in the data.

\subsubsection{Attention with prior}

Attention with prior enhances the standard self-attention mechanism by incorporating external or pre-existing knowledge into the attention distribution, rather than relying solely on query-key similarity. This prior can take various forms: it may encode structural information such as positional relationships (e.g., via trainable positional biases or Gaussian locality biases) \citep{Raffel2020}, reuse attention patterns from previous layers (acting as a form of residual or convolutional prior) \citep{Esearch2011}, or even serve as a task-specific adapter in transfer-learning setups \citep{ying2021lazyformerselfattentionlazy}. In some cases, the prior can entirely replace the dynamically generated attention—for example, using a fixed uniform or Gaussian distribution—which simplifies computation and can improve efficiency. Experiments show that the resulting model remains effective while being much more efficient to compute \citep{Lin2022}.

\subsubsection{Rationale}

While these variants significantly improve atttention-based models efficiency while preserving or even enhancing performance, they often introduce additional complexity in terms of implementation and hyperparameter tuning. Moreover, many of these methods still rely on the fundamental formulation of the standard attention mechanism, which may limit their ability to fully exploit the potential of alternative mathematical formulations, and even more, they enhance performance but lose the relationship inference power of the original attention mechanism. This motivates our exploration into a novel exponentiation-based attention mechanism that seeks to redefine the attention operation itself, aiming for a more parameter-efficient and theoretically grounded approach. 

\section{Methodology: Exponentiation of Non-Square Matrices}

The classical matrix exponential, defined for a square matrix $\A \in \mathbb{R}^{n \times n}$ via the Taylor series:

\begin{equation}
  \label{eq:matrix-exponential}
  \exp(\A) = \I{m}{n} + \sum_{k=1}^{\infty} \frac{\A^k}{k!},
\end{equation}

is inapplicable to non-square matrices that appear in attention due to the lack of a formal non-square matrix power function mechanism. To bridge this gap, we propose a formal definition for the exponential of a non-square matrix product. We hypothesize that such a definition can encapsulate the essence of the attention mechanism's exponential weighting, potentially leading to more parameter-efficient models. \\

Let $\A \in \mathbb{R}^{m \times n}$ and $\B \in \mathbb{R}^{n \times p}$. We seek to define an operation $\exp(\A \B^T)$ that retains the expressive power of the attention mechanism. A prerequisite for this is the definition of a custom product operation ($\circledast$) for non-square matrices that enables the construction of a meaningful power series. Let $\A, \B, \C \in \mathbb{R}^{m \times n}$. The proposed product operation should ideally satisfy the following algebraic properties:

\begin{enumerate}
  \item \textbf{Associativity:} $(\A \circledast \B) \circledast \C = \A \circledast (\B \circledast \C)$.
  \item \textbf{Identity Element:} There exists an element $\I{m}{n}$ such that $\I{m}{n} \circledast \A = \A$.
  \item \textbf{Null Element:} There exists an element $\mathbf{0}$ such that $\0{m}{n} \circledast \A = \0{m}{n}$.
\end{enumerate}

Hadamard product ($\odot$) is a well-known element-wise multiplication operation for matrices of the same dimensions; however, it lacks of elements recombination power of standard matrix multiplication. To address this, we propose a novel product operation $\circledast$ for non-square matrices that combines the element-wise multiplication with a summation over the rows, effectively allowing for a richer interaction between the elements of the matrices involved. The development of such an operation is the primary theoretical contribution of this work, forming the basis for our proposed exponentiation-based attention mechanism. Thus:

\begin{definition}
  Given $\A , \B \in \vectorspace{m}{n}$, we define the product operation $\circledast  : \vectorspace{m}{n} \to \vectorspace{m}{n}$ as a block matrix given by:

  \begin{equation}
    \label{eq:non-square-product}
    \A \circledast \B \defeq \sum_{k=1}^{n} \left[ \begin{matrix}
      \A_{k} \odot \B_{1} \\
      \vdots \\
      \A_{k} \odot \B_{m} \\
    \end{matrix} \right]
  \end{equation}

  where $\A_{i}$ denotes the $i$-th row of matrix $\A$, and $\odot$ represents the element-wise multiplication or Hadamard product.
\end{definition}

\begin{remark}
The product operation $\circledast$ defined in Equation~\ref{eq:non-square-product} possesses several key algebraic properties. It is associative, meaning $(\A \circledast \B) \circledast \C = \A \circledast (\B \circledast \C)$ for all $\A, \B, \C \in \mathbb{R}^{m \times n}$. Furthermore, there exists a identity element $\I{m}{n}$ such that $\I{m}{n} \circledast \A = \A$ for any $\A$, and a null element $\mathbf{0} \in \mathbb{R}^{m \times n}$ such that $\mathbf{0} \circledast \A = \mathbf{0}$. These properties are fundamental for the subsequent definition of a matrix exponential based on this product.
\end{remark}

\begin{proposition}
  The product operation $\circledast$ defined in Equation~\ref{eq:non-square-product} is associative, has a left identity element $\I{m}{n}$, and a null element $\0{m}{n}$. That is:
  \begin{enumerate}
    \item[(i)] \textbf{Associativity:} For $\A, \B, \C \in \vectorspace{m}{n}$, $(\A \circledast \B) \circledast \C =  \A \circledast (\B \circledast \C)$.
    \item[(ii)] \textbf{Identity Element:} Let $\A \in \vectorspace{m}{n}$ and $\I{m}{n}$ be the non-square identity matrix, then: $\I{m}{n} \circledast \A = \A$.
    \item[(iii)] \textbf{Null Element:} Let $\A \in \vectorspace{m}{n}$ and $\0{m}{n}$ be the zero non-square matrix, then: $\0={m}{n} \circledast \A = \0{m}{n}$.
  \end{enumerate}
\end{proposition}

\begin{proof}
  \begin{enumerate}
    \item[(i)] \textbf{Associativity:} Let $\A, \B, \C \in \vectorspace{m}{n}$. Then:
      \begin{align*}
        (\A \circledast \B) \circledast \C &
        = \sum_{p=1}^{n} \left[ \begin{matrix}
          \A_{p} \odot \B_{1} \\
          \vdots \\
          \A_{p} \odot \B_{m} \\
        \end{matrix} \right] \circledast \C = \left[ \begin{matrix}
          \sum_{p=1}^{n} \A_{p} \odot \B_{1} \\
          \vdots \\
          \sum_{p=1}^{n} \A_{p} \odot \B_{m} \\
        \end{matrix} \right] \circledast \C \\
      \end{align*}  
      \begin{align*} 
        (\A \circledast \B) \circledast \C &
        = \sum_{q=1}^{n} \left[ \begin{matrix}
          \left(\sum_{p=1}^{n} \A_{p} \odot \B_{q}\right) \odot C_1 \\
          \vdots \\
          \left(\sum_{p=1}^{n} \A_{p} \odot \B_{q}\right) \odot C_m\\
        \end{matrix} \right] \\
        & = \sum_{p=1}^{n} \left[ \begin{matrix}
          \A_{p} \odot \left(\sum_{q=1}^{n}  \B_{q} \odot C_1 \right) \\
          \vdots \\
          \A_{p} \odot \left(\sum_{q=1}^{n} \B_{q} \odot C_m \right) \\
        \end{matrix} \right] \\
        & = \A \circledast \sum_{p=1}^{n} \left[ \begin{matrix}
          \left(\sum_{q=1}^{n}  \B_{q} \odot C_1 \right) \\
          \vdots \\
          \left(\sum_{q=1}^{n} \B_{q} \odot C_m \right)
        \end{matrix} \right] = \A \circledast (\B \circledast \C) \\
      \end{align*}
    \item[(ii)] \textbf{Identity Element:} Let $\A \in \vectorspace{m}{n}$ and $\I{m}{n}$ be the non-square identity matrix such that: 
      \[
        \left(\I{m}{n}\right)_{i,j} = \begin{cases}
          1 & \text{if } i = j \\
          0 & \text{otherwise}
        \end{cases}
      \]
      Then:
      \begin{align*}
        \I{m}{n} \circledast \A &
        = \sum_{k=1}^{n} \left[ \begin{matrix}
          \left(\I{m}{n}\right)_{k} \odot A_1 \\
          \vdots \\
          \left(\I{m}{n}\right)_{k} \odot A_m \\
        \end{matrix} \right] = \left[ \begin{matrix}
          \sum_{k=1}^{n} \left(\I{m}{n}\right)_{k} \odot A_1 \\
          \vdots \\
          \sum_{k=1}^{n} \left(\I{m}{n}\right)_{k} \odot A_m \\
        \end{matrix} \right]
      \end{align*}
       Observe that $\left(\I{m}{n} \circledast \A\right)_{i,j} = \sum_{k=1}^{n} \left(\I{m}{n}\right)_{k, j} \odot A_{i,j} = A_{i,j}$ since only the term where $k = j$ contributes to the sum. Therefore, $\I{m}{n} \circledast \A = \A$.
    \item[(iii)] \textbf{Null Element:} Let $\A \in \vectorspace{m}{n}$ and $\0{m}{n}$ be the zero non-square matrix ($\left(\0{m}{n}\right)_{i,j}= 0$). Then $\left(\0{m}{n} \circledast \A\right)_{i,j} = 0$ for all $i, j$ since each term in the sum involves multiplication by zero. Thus, $\0{m}{n} \circledast \A = \0{m}{n}$.
  \end{enumerate}
\end{proof}

Defining the power of a non-square matrix under the $\circledast$ operation allows us to extend the concept of the matrix exponential to non-square matrices. It follows.

\begin{definition}
  Let $\A \in \vectorspace{m}{n}$, denote $\A^n$ as the $n$-th power of $\A$ under the $\circledast$ operation, defined recursively as:
  \begin{equation*}
    \A^n \defeq \begin{cases}
      \I{m}{n} & \text{if } n = 0,
      \\
      \A^{n-1} \circledast \A & \text{if } n \geq 1.
    \end{cases}
  \end{equation*}
\end{definition}

\begin{definition}
  Given $\A \in \vectorspace{m}{n}$, we define the exponential of the non-square matrix $\A$ as:
  \begin{equation}
    \label{eq:non-square-exponential}
    \exp(\A) \defeq \I{m}{n} + \sum_{k=1}^{\infty} \frac{\A^k}{k!},
  \end{equation}
  where $\A^k$ is defined using the $\circledast$ operation.
\end{definition}

For practical implementation, we approximate the infinite series in Equation~\ref{eq:non-square-exponential} by truncating it at a finite number of terms $K$:

\begin{equation}
  \label{eq:truncated-non-square-exponential}
  \exp(\A) \approx \I{m}{n} + \sum_{k=1}^{K} \frac{\A^k}{k!}.
\end{equation}

\subsection{Non-Square Exponential Attention Mechanism}

Building upon the defined non-square matrix exponential, we propose a novel attention mechanism that leverages this operation to capture high-order interactions in the input data. Similarly to the linearized attention mechanism, we aim to capture complex relationships by preserving a radial basis function kernel structure ($\expo{{-\Vert{x}\Vert^2\lambda}}$), and preserving attention's exponential nature and dimensionality.

The Non-Square Exponential Attention (NSEA) mechanism is defined as follows: based on the radial structure of multi-head attention, the radial basis functions, and the exponential form of attention described earlier. The NSEA for a given input matrix $\X \in \mathbb{R}^{m \times n}$ is defined as:

\begin{equation}
  \label{eq:nsea}
  H \defeq \exp\left( (\X \W + \B)^2 \mathbf{\Lambda} \right)
\end{equation}


where $\W \in \mathbb{R}^{n \times d_k}$,  $\mathbf{\Lambda} \in \mathbb{R}^{d_k \times d_v}$. Where \(\mathbf{\Lambda}\) is a learnable diagonal matrix that scales the contributions of each dimension in the transformed space, allowing the model to adaptively focus on different aspects of the input features. The bias term \(\B \in \mathbb{R}^{m \times d_k}\) is included to provide additional flexibility in the transformation, enabling the model to better capture complex patterns in the data. The output \(H \in \mathbb{R}^{m \times d_v}\) represents the attention-weighted representations of the input tokens, where \(d_v\) is the dimensionality of the value vectors. \\

\section{Experiments and Results}

To evaluate the efficiency of the NSEA mechanism, we compare its performance against traditional multi-head attention. 

\subsubsection{Computation Effeciency}

We analyszed the computation effeciency of NSEA compared against Multi-head attention (MHA), by measuring the FLOPs taken for forward passes on varying vocab sizes and batch length for a random input sequence of lengh $16384 = (4096 \times 4)$, an wmbedding size of $2048$ (DeeepSeek's maximm sequence length and embedding size \citep{deepseekai2025deepseekv3technicalreport}), and a $4$-degree Taylor aproxximation on a computer with $24$ GB of RAM with a speed of $4800 MT/s$, and a $13^{th}$ Generation Intel(R) Core(TM) i5-3420H. Figure~\ref{fig:computation_efficiency} illustrates the testing setup.

\begin{figure}[h!]
    \centering
    \includegraphics[width=0.7\textwidth]{images/computation_efficiency_setup.png}
    \caption{Computation Efficiency Testing Setup}
    \label{fig:computation_efficiency}
\end{figure}

\begin{table}[h]
    \centering
    \resizebox{1.0\textwidth}{!}{
    \begin{tabular}{ccccc}
        \hline
        \textbf{Model} & \textbf{Vocab} & \textbf{Total} & \textbf{Per Batch} & 
        \textbf{Per Token} \\
        & \textbf{Size} & \textbf{Batches} & \textbf{Processing Time} &
        \textbf{Processing Time} \\
        & & & $(\mu{s})$ & $(\text{ps})$ \\
        \hline
        HEA & 128 & 32 & 9617.4702 $\pm$ 2.7330 & 18.7841 $\pm$ 0.0000 \\
        MHA & 128 & 32 & 8884.5193 $\pm$ 1.2986 & 17.3526 $\pm$ 0.0000 \\
        NSEA & 128 & 32 & 10298.1851 $\pm$ 4.1715 & 20.1136 $\pm$ 0.0000 \\
        \hline
        MHA & 256 & 32 & 9206.8017 $\pm$ 1.7937 & 17.9820 $\pm$ 0.0000 \\
        NSEA & 256 & 32 & 10066.7030 $\pm$ 1.2845 & 19.6615 $\pm$ 0.0000 \\
        HEA & 256 & 32 & 10052.8225 $\pm$ 2.0561 & 19.6344 $\pm$ 0.0000 \\
        \hline
        HEA & 512 & 32 & 9982.5040 $\pm$ 1.6734 & 19.4971 $\pm$ 0.0000 \\
        MHA & 512 & 32 & 10245.0252 $\pm$ 2.3697 & 20.0098 $\pm$ 0.0000 \\
        NSEA & 512 & 32 & 10054.4989 $\pm$ 2.1359 & 19.6377 $\pm$ 0.0000 \\
        \hline
        NSEA & 1024 & 32 & 10043.1740 $\pm$ 1.2447 & 19.6156 $\pm$ 0.0000 \\
        MHA & 1024 & 32 & 10199.7927 $\pm$ 2.3696 & 19.9215 $\pm$ 0.0000 \\
        HEA & 1024 & 32 & 10381.7210 $\pm$ 1.5100 & 20.2768 $\pm$ 0.0000 \\
        \hline
        MHA & 2048 & 32 & 9993.7469 $\pm$ 1.5803 & 19.5190 $\pm$ 0.0000 \\
        HEA & 2048 & 32 & 10188.0133 $\pm$ 1.1538 & 19.8985 $\pm$ 0.0000 \\
        NSEA & 2048 & 32 & 10974.4519 $\pm$ 2.5388 & 21.4345 $\pm$ 0.0000 \\
        \hline
        HEA & 4096 & 32 & 13237.5732 $\pm$ 25.6088 & 25.8546 $\pm$ 0.0001 \\
        MHA & 4096 & 32 & 9705.0592 $\pm$ 2.1649 & 18.9552 $\pm$ 0.0000 \\
        NSEA & 4096 & 32 & 10204.9112 $\pm$ 2.7498 & 19.9315 $\pm$ 0.0000 \\
        \hline
        NSEA & 8192 & 32 & 12324.4748 $\pm$ 16.6725 & 24.0712 $\pm$ 0.0001 \\
        MHA & 8192 & 32 & 9781.6363 $\pm$ 1.1224 & 19.1048 $\pm$ 0.0000 \\
        HEA & 8192 & 32 & 10015.7112 $\pm$ 1.5895 & 19.5619 $\pm$ 0.0000 \\
        \hline
        NSEA & 16384 & 32 & 10165.3337 $\pm$ 1.7920 & 19.8542 $\pm$ 0.0000 \\
        MHA & 16384 & 32 & 9987.5554 $\pm$ 20.4950 & 19.5069 $\pm$ 0.0001 \\
        HEA & 16384 & 32 & 10132.6033 $\pm$ 1.8536 & 19.7902 $\pm$ 0.0000 \\
        \hline
        NSEA & 32768 & 32 & 9633.1462 $\pm$ 1.4925 & 18.8147 $\pm$ 0.0000 \\
        MHA & 32768 & 32 & 9628.4226 $\pm$ 1.0798 & 18.8055 $\pm$ 0.0000 \\
        HEA & 32768 & 32 & 9892.4562 $\pm$ 2.2805 & 19.3212 $\pm$ 0.0000 \\
        \hline
        HEA & 65536 & 32 & 9977.2811 $\pm$ 1.0690 & 19.4869 $\pm$ 0.0000 \\
        NSEA & 65536 & 32 & 9751.4838 $\pm$ 1.6471 & 19.0459 $\pm$ 0.0000 \\
        MHA & 65536 & 32 & 9590.4022 $\pm$ 1.6857 & 18.7313 $\pm$ 0.0000 \\
        \hline
        HEA & 131072 & 32 & 10188.2517 $\pm$ 1.8249 & 19.8989 $\pm$ 0.0000 \\
        MHA & 131072 & 32 & 10212.2724 $\pm$ 1.1477 & 19.9458 $\pm$ 0.0000 \\
        NSEA & 131072 & 32 & 10047.2271 $\pm$ 1.6459 & 19.6235 $\pm$ 0.0000 \\
        \hline
    \end{tabular}}
    \caption{Computation efficiency comparison between Multi-head Attention (MHA),
    Non-Square Exponential Attention (NSEA), and adamard-based Exponential Attention (HEA)
    on a random input of length $16384$.
    }
    \label{tab:computation_efficiency}
\end{table}


Table~\ref{tab:computation_efficiency} indicates that HEA outperforms MHA. Results prove that NSEA is significantly more efficient than MHA in terms of FLOPs, with a compression ratio of approximately $257$ times. This efficiency gain is attributed to the reduced computational complexity of the non-square exponential operation compared to the traditional attention mechanism, which involves more intensive matrix multiplications and softmax operations.


\end{document}