\section{Experiments and Results}

To evaluate the efficiency of the NSEA mechanism, we compare its performance against traditional multi-head attention. 

\subsubsection{Computation Effeciency}

We analyszed the computation effeciency of NSEA compared against Multi-head attention (MHA), by measuring the FLOPs taken for forward passes on varying vocab sizes and batch length for a random input sequence of lengh $16384 = (4096 \times 4)$, an wmbedding size of $2048$ (DeeepSeek's maximm sequence length and embedding size \citep{deepseekai2025deepseekv3technicalreport}), and a $4$-degree Taylor aproxximation on a computer with $24$ GB of RAM with a speed of $4800 MT/s$, and a $13^{th}$ Generation Intel(R) Core(TM) i5-3420H. Figure~\ref{fig:computation_efficiency} illustrates the testing setup.

\begin{figure}[h!]
    \centering
    \includegraphics[width=0.7\textwidth]{images/computation_efficiency_setup.png}
    \caption{Computation Efficiency Testing Setup}
    \label{fig:computation_efficiency}
\end{figure}

\begin{table}[h]
    \centering
    \resizebox{1.0\textwidth}{!}{
    \begin{tabular}{ccccc}
        \hline
        \textbf{Model} & \textbf{Vocab} & \textbf{Total} & \textbf{Per Batch} & 
        \textbf{Per Token} \\
        & \textbf{Size} & \textbf{Batches} & \textbf{Processing Time} &
        \textbf{Processing Time} \\
        & & & $(\mu{s})$ & $(\text{ps})$ \\
        \hline
        HEA & 128 & 32 & 9617.4702 $\pm$ 2.7330 & 18.7841 $\pm$ 0.0000 \\
        MHA & 128 & 32 & 8884.5193 $\pm$ 1.2986 & 17.3526 $\pm$ 0.0000 \\
        NSEA & 128 & 32 & 10298.1851 $\pm$ 4.1715 & 20.1136 $\pm$ 0.0000 \\
        \hline
        MHA & 256 & 32 & 9206.8017 $\pm$ 1.7937 & 17.9820 $\pm$ 0.0000 \\
        NSEA & 256 & 32 & 10066.7030 $\pm$ 1.2845 & 19.6615 $\pm$ 0.0000 \\
        HEA & 256 & 32 & 10052.8225 $\pm$ 2.0561 & 19.6344 $\pm$ 0.0000 \\
        \hline
        HEA & 512 & 32 & 9982.5040 $\pm$ 1.6734 & 19.4971 $\pm$ 0.0000 \\
        MHA & 512 & 32 & 10245.0252 $\pm$ 2.3697 & 20.0098 $\pm$ 0.0000 \\
        NSEA & 512 & 32 & 10054.4989 $\pm$ 2.1359 & 19.6377 $\pm$ 0.0000 \\
        \hline
        NSEA & 1024 & 32 & 10043.1740 $\pm$ 1.2447 & 19.6156 $\pm$ 0.0000 \\
        MHA & 1024 & 32 & 10199.7927 $\pm$ 2.3696 & 19.9215 $\pm$ 0.0000 \\
        HEA & 1024 & 32 & 10381.7210 $\pm$ 1.5100 & 20.2768 $\pm$ 0.0000 \\
        \hline
        MHA & 2048 & 32 & 9993.7469 $\pm$ 1.5803 & 19.5190 $\pm$ 0.0000 \\
        HEA & 2048 & 32 & 10188.0133 $\pm$ 1.1538 & 19.8985 $\pm$ 0.0000 \\
        NSEA & 2048 & 32 & 10974.4519 $\pm$ 2.5388 & 21.4345 $\pm$ 0.0000 \\
        \hline
        HEA & 4096 & 32 & 13237.5732 $\pm$ 25.6088 & 25.8546 $\pm$ 0.0001 \\
        MHA & 4096 & 32 & 9705.0592 $\pm$ 2.1649 & 18.9552 $\pm$ 0.0000 \\
        NSEA & 4096 & 32 & 10204.9112 $\pm$ 2.7498 & 19.9315 $\pm$ 0.0000 \\
        \hline
        NSEA & 8192 & 32 & 12324.4748 $\pm$ 16.6725 & 24.0712 $\pm$ 0.0001 \\
        MHA & 8192 & 32 & 9781.6363 $\pm$ 1.1224 & 19.1048 $\pm$ 0.0000 \\
        HEA & 8192 & 32 & 10015.7112 $\pm$ 1.5895 & 19.5619 $\pm$ 0.0000 \\
        \hline
        NSEA & 16384 & 32 & 10165.3337 $\pm$ 1.7920 & 19.8542 $\pm$ 0.0000 \\
        MHA & 16384 & 32 & 9987.5554 $\pm$ 20.4950 & 19.5069 $\pm$ 0.0001 \\
        HEA & 16384 & 32 & 10132.6033 $\pm$ 1.8536 & 19.7902 $\pm$ 0.0000 \\
        \hline
        NSEA & 32768 & 32 & 9633.1462 $\pm$ 1.4925 & 18.8147 $\pm$ 0.0000 \\
        MHA & 32768 & 32 & 9628.4226 $\pm$ 1.0798 & 18.8055 $\pm$ 0.0000 \\
        HEA & 32768 & 32 & 9892.4562 $\pm$ 2.2805 & 19.3212 $\pm$ 0.0000 \\
        \hline
        HEA & 65536 & 32 & 9977.2811 $\pm$ 1.0690 & 19.4869 $\pm$ 0.0000 \\
        NSEA & 65536 & 32 & 9751.4838 $\pm$ 1.6471 & 19.0459 $\pm$ 0.0000 \\
        MHA & 65536 & 32 & 9590.4022 $\pm$ 1.6857 & 18.7313 $\pm$ 0.0000 \\
        \hline
        HEA & 131072 & 32 & 10188.2517 $\pm$ 1.8249 & 19.8989 $\pm$ 0.0000 \\
        MHA & 131072 & 32 & 10212.2724 $\pm$ 1.1477 & 19.9458 $\pm$ 0.0000 \\
        NSEA & 131072 & 32 & 10047.2271 $\pm$ 1.6459 & 19.6235 $\pm$ 0.0000 \\
        \hline
    \end{tabular}}
    \caption{Computation efficiency comparison between Multi-head Attention (MHA),
    Non-Square Exponential Attention (NSEA), and adamard-based Exponential Attention (HEA)
    on a random input of length $16384$.
    }
    \label{tab:computation_efficiency}
\end{table}


Table~\ref{tab:computation_efficiency} indicates that HEA outperforms MHA. Results prove that NSEA is significantly more efficient than MHA in terms of FLOPs, with a compression ratio of approximately $257$ times. This efficiency gain is attributed to the reduced computational complexity of the non-square exponential operation compared to the traditional attention mechanism, which involves more intensive matrix multiplications and softmax operations.
